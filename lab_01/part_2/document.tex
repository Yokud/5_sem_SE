\documentclass[a4paper,12pt]{extreport}

\usepackage{cmap}
\usepackage[T2A]{fontenc}
\usepackage[utf8]{inputenc}
\usepackage[english,russian]{babel}

\usepackage{amsmath}

\usepackage{geometry}
\geometry{left=30mm}
\geometry{right=15mm}
\geometry{top=20mm}
\geometry{bottom=20mm}

\usepackage{titlesec}
\titleformat{\section}
{\normalsize\bfseries}
{\thesection}
{1em}{}
\titlespacing*{\chapter}{0pt}{-30pt}{8pt}
\titlespacing*{\section}{\parindent}{*4}{*4}
\titlespacing*{\subsection}{\parindent}{*4}{*4}

\usepackage{setspace}
\onehalfspacing

\frenchspacing
\usepackage{indentfirst}

\usepackage{titlesec}
\titleformat{\chapter}{\LARGE\bfseries}{\thechapter}{18pt}{\LARGE\bfseries}
\titleformat{\section}{\Large\bfseries}{\thesection}{18pt}{\Large\bfseries}

\usepackage{xcolor}


\usepackage{pgfplots}
\usetikzlibrary{datavisualization}
\usetikzlibrary{datavisualization.formats.functions}

\usepackage{graphicx}
\newcommand{\img}[3] {
	\begin{figure}[h]
		\center{\includegraphics[height=#1]{assets/img/#2}}
		\caption{#3}
		\label{img:#2}
	\end{figure}
}

\newcommand{\imgw}[3] {
	\begin{figure}[h]
		\center{\includegraphics[width=#1]{assets/img/#2}}
		\caption{#3}
		\label{img:#2}
	\end{figure}
}

\usepackage[justification=centering]{caption}
\usepackage[unicode,pdftex]{hyperref}
\hypersetup{hidelinks}
\newcommand{\code}[1]{\texttt{#1}}
\usepackage{icomma}
\usepackage{csvsimple}
\usepackage{svg}

\newcommand\Tstrut{\rule{0pt}{2.6ex}}       % "top" strut
\newcommand\Bstrut{\rule[-0.9ex]{0pt}{0pt}} % "bottom" strut
\newcommand{\TBstrut}{\Tstrut\Bstrut} % top&bottom struts

\begin{document}
\newgeometry{pdftex, left=2cm, right=2cm, top=2.5cm, bottom=2.5cm}
\fontsize{12pt}{12pt}\selectfont
\noindent \begin{minipage}{0.15\textwidth}
	\includegraphics[width=\linewidth]{b_logo.jpg}
\end{minipage}
\noindent\begin{minipage}{0.9\textwidth}\centering
	\textbf{Министерство науки и высшего образования Российской Федерации}\\
	\textbf{Федеральное государственное бюджетное образовательное учреждение высшего образования}\\
	\textbf{«Московский государственный технический университет имени Н.Э.~Баумана}\\
	\textbf{(национальный исследовательский университет)»}\\
	\textbf{(МГТУ им. Н.Э.~Баумана)}
\end{minipage}

\noindent\rule{18cm}{3pt}
\newline\newline
\noindent ФАКУЛЬТЕТ $\underline{\text{«Информатика и системы управления»}}$ \newline\newline
\noindent КАФЕДРА $\underline{\text{«Программное обеспечение ЭВМ и информационные технологии»}}$\newline\newline\newline\newline\newline\newline\newline


\begin{center}
	\Large\textbf{Отчет по лабораторной работе №1 (часть 2) по дисциплине <<Операционные системы>>}
\end{center}

""\newline\newline

\noindent\textbf{Тема} $\underline{\text{Прерывания таймера в Windows и UNIX~~~~~~~}}$\newline\newline
\noindent\textbf{Студент} $\underline{\text{Малышев И. А.~~~~~~~~~~~~~~~~~~~~~~~~~~~~~~~~~~~~}}$\newline\newline
\noindent\textbf{Группа} $\underline{\text{ИУ7-51Б~~~~~~~~~~~~~~~~~~~~~~~~~~~~~~~~~~~~~~~~~~~~~~~}}$\newline\newline
\noindent\textbf{Преподаватель} $\underline{\text{Рязанова Н. Ю.~~~~~~~~~~~~~~~~~~~~~~~~~}}$\newline

\begin{center}
	\vfill
	Москва~---~\the\year
	~г.
\end{center}
\restoregeometry

\newpage

\chapter{Функции обработчика прерывания от системного таймера в защищённом режиме}

\section{Windows-системы}

Функции обработчика прерывания от системного таймера \textbf{по тику}:
\begin{itemize}
	\item инкремент счётчика системного времени;
	\item декремент остатка кванта текущего потока;
	\item декремент счётчиков времени отложенных задач;
	\item если активен механизм профилирования ядра, инициализирует отложенный вызов обработчика ловушки ядра путём постановки объекта в \textbf{DPC}-очередь: обработчик ловушки профилирования регистрирует адрес команды, выполнявшейся на момент прерывания.
\end{itemize}

Функции обработчика прерывания от системного таймера \textbf{по главному тику}:
\begin{itemize}
	\item инициализация диспетчера настройки баланса путём сбрасывания объекта <<событие>>, на котором он ожидает.
\end{itemize}

Функции обработчика прерывания от системного таймера \textbf{по кванту}:
\begin{itemize}
	\item инициализация диспетчеризации потоков путём добавления соответствующего объекта в \textbf{DPC}-очередь.
\end{itemize}


\section{UNIX-системы}

Функции обработчика прерывания от системного таймера \textbf{по тику}:
\begin{itemize}
	\item инкремент счётчика тиков аппаратного таймера;
	\item декремент кванта текущего потока;
	\item обновление статистики использования процессора текущим процессом - инкремент поля \textbf{p\_cpu} дескриптора текущего процесса до максимального значения 127;
	\item инкремент часов и других таймеров системы;
	\item 
\end{itemize}

Функции обработчика прерывания от системного таймера \textbf{по главному тику}:
\begin{itemize}
	\item 
	\item 
	\item 
\end{itemize}

Функции обработчика прерывания от системного таймера \textbf{по кванту}:
\begin{itemize}
	\item отправка текущему процессу сигнала \textbf{SIGXCPU}, если тот превысил выделенную ему квоту использования процессора.
\end{itemize}

\end{document}